\documentclass[aspectratio=169,colorlinks]{beamer}
\usetheme{Boadilla} % plainest one with slide number footer

% generic packages

\usepackage[utf8]{inputenc}
\usepackage[english]{babel}
\usepackage{ulem}
\usepackage{hyperref}

% about the presentation
\title{Attacks on CoAP}
\subtitle{\texttt{draft-ietf-core-attacks-on-coap}}
\author{John~Preuß~Mattsson, John~Fornehed, Göran~Selander, Francesca~Palombini, \textit{Christian~Amsüss}}
\date{IETF117 San Francisco, CoRE, 2023-07-25}

% used commands

\usepackage{verbatim}

\newcommand{\rfc}[1]{\href{https://datatracker.ietf.org/doc/html/rfc#1}{RFC~#1}}
\newcommand{\ietfdraft}[1]{\href{https://datatracker.ietf.org/doc/draft-#1/}{draft #1}}

% attach self

\usepackage{embedfile}
\embedfile{\jobname.tex}
\embedfile{Makefile}

\begin{document}

\frame{\titlepage}

\begin{frame}{History: IETF99}
	\fbox{\includegraphics[page=98,width=0.6\pagewidth]{slides-99-core-consolidated-slides-06.pdf}}
\end{frame}

\begin{frame}{History: IETF99}
	\fbox{\includegraphics[page=99,width=0.6\pagewidth]{slides-99-core-consolidated-slides-06.pdf}}

	\texttt{repeat-request-tag} became \texttt{echo-request-tag}, and eventually \rfc{9175}

	\texttt{coap-actuators} became \texttt{attacks-on-coap}.
\end{frame}

\begin{frame}{Content}\large
	``Here is what can go wrong
	in the presence of an on-path attacker
	that may even guess plain text
	but does not break the cryptographic primitives,
	and there (eg. \rfc{9175}) is how to avoid it.''

	\vspace{2cm}

	Except, we're may not be done :'-(
\end{frame}

\begin{frame}{Return of the Request Fragment Rearrangement}\large
	\framesubtitle{Praphrasing dialogue between Jon Shallow and Christian Amsüss}
	\begin{itemize}
		\item[J] That can also happen when there is no Block1 phase.
		\item[C] No, because then it's concurrent requests, and those don't work without Request-Tag anyway. In the Block2 phase, the request payload is repeated.
		\item[J] The problem is still there.
		\item[C] \textit{reads \rfc{7959} and blushes.}
	\end{itemize}

	\bigskip

	Turns out, the request payload is very much not repeated.
\end{frame}

\begin{frame}[fragile]{Return of the Request Fragment Rearrangement}\small
	\framesubtitle{Example}
	\begin{verbatim}
   Client   Foe   Server
      |      |      |
      +------X      |    POST "request" T:1 { "offset":0, "length":2000}
      |      |      |
      +------------->    POST "request" T:2 { "offset":4000, "length":2000}
      |      |      |
      |      @------>    POST "request" T:1 { "offset":0, "length":2000}
      |      |      |
      <-------------+    2.04 T:2 Block2:0/1/1024 { data containing 4000:1024 }
      |      |      |
      <-------------+    2.04 T:1 Block2:0/1/1024 { data containing 0:1024 }
      |      |      |
      +------------->    POST "request" T:3 Block2:0/_/1024
                         server - is this continuation of request using T:1 or T:2 ?
      |      |      |
      <-------------+    2.04 T:3 Block2:1/_/1024 { data containing 1024:2000 }
                         Was this the expected data ?
      |      |      |
	\end{verbatim}
\end{frame}

\begin{frame}{Possible solution A: Use Request-Tag more often}\large
	Old: ``When a client fragments a request body into multiple message payloads [ the body integrity is not protected ]. To gain that protection, use the Request-Tag mechanism as follows:''

	\bigskip

	New: ``\tiny{}When a client fragments a request body into multiple message payloads\large{}, or on requests that are matchable to requests the client may have produced and is prepared to accept fragmented responses for (even if the request body is small or empty), \tiny [ the body integrity is not protected ]. To gain that protection, use the Request-Tag mechanism a follows:''

	\large
	\bigskip

	Could be short document, updates \rfc{9175} (maybe also \rfc{9177}).
\end{frame}

\begin{frame}{Practical Impact of using Request-Tag more often}\large
	When using OSCORE: Negligible.

	Constrained OSCORE implementations can always use the absent Request-Tag option
	as long as they spot some conditions after which they bump their sender sequence number.

	\bigskip

	When using DTLS:
	Almost all requests that could have requests that could have resquest and response bodies
	will need a Request-Tag, typically a counter.

	\ldots But they already tolerate having tokens that practically need to be counting up. How large is the impact really?
\end{frame}

\begin{frame}{Possible solution B: Allow server to declare Request-Tag}\large
	\framesubtitle{Suggestion from Jon}

	Would update \rfc{9175} to allow use in requests.

	Shifts logic and state tracking to the server.
\end{frame}

\begin{frame}{Possible solution C: Tighten ETag}\large
	Add some small requirements around ETags
	to ensure they prevent the attack.

	\bigskip

	``\ldots are necessary even if the represention does not change over time but is a function of the request payload\ldots''

	``\ldots ETags must be unique to the body even across request payloads \ldots''

	\bigskip

	Could be short document, updates \rfc{7252} or \rfc{7959} (maybe also \rfc{9175}).
\end{frame}

\begin{frame}{Roadmap}\Large
	\begin{enumerate}
		\item Tune example, decide whether single example covers it well.
		\item Write text for proposed solutions.
		\item Pick best solution.
		\item Ask WG to adopt solution.
		\item Continue with this document when we can point to solution.
	\end{enumerate}
\end{frame}

\begin{frame}{Thanks}\Large
	Comments?

	\bigskip

	Questions?
\end{frame}

\end{document}
